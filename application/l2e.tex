\documentclass[11pt]{lucs-art}
\usepackage{xspace}
\newcommand{\tested}{TeStED\xspace}
\newcommand{\ltoe}{LEARN{\kern 1pt}2\emph{Engage}\xspace}
\usepackage[compact]{titlesec}
\usepackage{enumitem}
\setlist{itemsep=-2pt}

\title{\ltoe: A self-assessment dashboard to improve student engagement}
\author{Iain Phillips\and Dominik Freydenberger}
\date{January 2020}
\begin{document}
\maketitle
\section{Summary}

University students are much more in control of their learning than in
previous education.  They control when, how and on what they work.
This means there is no one to coach them through pointing out that
they have fallen behind.  Tools to measure attendance and submission
are natural parts of our systems.  However, these are poor
measurements of engagement due to their timing.  This project aims to
provide students with an improved engagement measure, using another
aspect of their learning: the material on LEARN\@.  This project will create  a simple, interactive \emph{engagement dashboard} for students.

\section{Resources}

\begin{itemize}
\item This is a lean, lightweight project.
\item Drs Phillips and Freydenberger will contribute at most 0.05FTE
  of their time each.
\item No reduction in teaching is expected for each of them.
\item No additional funding will be required.
\item The work may be linked with a student project in the 2020-21
  academic year.
\item We will require some IT Support to ensure security and safety of
  the stored dataset.
\item We will require access to LEARN logs.  IT Services have confirmed
  their support.
\end{itemize}

\section{Description}

Technically the project will take LEARN logs and pass them through a
series of processes to provide a dashboard (referred to as an
\emph{engagement scene}).  The scene will provide a student with a
view of their engagement with their learning, hence: \ltoe.

LEARN doesn’t provide student with a convenient way to identify which
material they have looked at and which not, e.g.\ worksheets,
lectures, quizzes, etc.  Although these data are present in LEARN it
is in the form of log files rather than databases.  This project will
process such files to measure times of access of each document.
These new meta-data will be used to provide students with a simple
view of their current engagement with their studies.

The aim will be for students to engage with the studies well
before the normal trigger times---revision week and exams.

\subsection{Links to Strategy}

\begin{itemize}
\item \emph{A life-shaping student experience}.  Students are
  self-managed learners---the best students, who will become the best
  workers, are best at self-management.  This project's deliverable
  doesn't control students, but visualises their engagement.  We aim
  to empower the students to take control of their lives, not control
  it for them.
\item\emph{Higher Education in the 21st Century}.  The increase use of
  digital technology to guide and support learning leads to the
  increased need to monitor progress.  This can come from formative
  assessments, summative exams and coursework, but these are often too
  coarse grained and specific to final marks.  This tool will provide
  a more personalised and fine-grained measure of engagement.
\item\emph{Raising standards and aspirations}.  The excellence of our
  students is long established and our student experience is extremely
  well regarded.  This project will further enhance both by enabling
  earlier engagement and consequently better assessment results.
\end{itemize}

\subsection{Objectives}

\begin{itemize}
\item To improve student engagement and connection to their studies.
\item To make students aware of the resources they have ignored/missed
  and the volume of these.
\item To measure students personal views on their engagement with a
  set of learning materials. (see phase 2)
\end{itemize}
A search of log files of accesses to different resources will be used
to provide students with a simple red/amber/green view of their
engagement so far.

\subsection{Deliverables}
All work on this project will take place between the time of project
award and the end of Semester 1 2020/21.  If successful all students
and all staff will benefit from the outcomes of this project.
\begin{itemize}
\item A mechanism to define connections between LEARN resources,
  e.g.\ lectures slides, online quizzes, exercise sheet and solutions
  sheets for a particular week.
\item A prototype tool to grab and process logs from LEARN.
\item A prototype tool to visualise an \emph{engagement scene} for
  particular student at a particular time.
\end{itemize}

\subsection{Routes to impact and evaluation}

While the overall project could be large, we have followed the
principles of Occam's Razor and split the projects in to smaller
parts.  The project could have three or more phases:
\begin{enumerate}
\item Feasibility study---firstly understanding the available data and
  technologies to access the data and secondly providing some simple
  analysis.
\item User study---trial the outputs of the feasibility study with
  sample users (students, admin and academic staff) and receive
  feedback to increase the value of the summary data displayed and,
  hopefully, the student engagement.  Here we will also provide a
  simple interface for students to comment on their own engagement
  and employ AI so staff and students can learn the best ways to
  achieve the best results.  This will be a bigger project.
\item Roll out---Integrate the output with the students' dashboards on LEARN\@.  This will require working with IT Services' LEARN team.
\end{enumerate}
This project will address the first of these points.  We have a 3-stage project and this is only funding the first stage.
If successful, we will apply for a further proposal next year.  If
that is successful, phase 3 will follow, whether as a \tested
proposal or a full University project remains to be seen.

\subsection{Partners and Other Support}

\begin{itemize}
\item Rich Goodman of IT Services will provide access to appropriate
  LEARN logs.
\item If successful, we will approach Sammy Chester for additional
  support from CAP/OD.
\item We will engage with other academic staff for their comments
  during phase 1.
\end{itemize}







\end{document}